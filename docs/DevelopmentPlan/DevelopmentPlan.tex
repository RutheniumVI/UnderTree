\documentclass{article}

\usepackage{booktabs}
\usepackage{tabularx}

\title{Development Plan\\\progname}

\author{\authname}

\date{}

\input{../Comments}
%% Common Parts

\newcommand{\progname}{UnderTree} % PUT YOUR PROGRAM NAME HERE
\newcommand{\authname}{Team 22, Capstoners
\\ Palanichamy Veerash, palanicv
\\ Kannammalil Kevin, kannammk
\\ Qureshi Eesha, qureshe
\\ Ahmed Faiq, ahmedf46} % AUTHOR NAMES                  

\usepackage{hyperref}
    \hypersetup{colorlinks=true, linkcolor=blue, citecolor=blue, filecolor=blue,
                urlcolor=blue, unicode=false}
    \urlstyle{same}
                                


\begin{document}

\begin{table}[hp]
\caption{Revision History} \label{TblRevisionHistory}
\begin{tabularx}{\textwidth}{llX}
\toprule
\textbf{Date} & \textbf{Developer(s)} & \textbf{Change}\\
\midrule
Date1 & Name(s) & Description of changes\\
Date2 & Name(s) & Description of changes\\
... & ... & ...\\
\bottomrule
\end{tabularx}
\end{table}

\newpage

\maketitle

\wss{Put your introductory blurb here.}

\section{Team Meeting Plan}

\section{Team Communication Plan}

\section{Team Member Roles}

\section{Workflow Plan}

\begin{itemize}
	\item How will you be using git, including branches, pull request, etc.?
	\item How will you be managing issues, including template issues, issue
	classificaiton, etc.?
\end{itemize}

\section{Proof of Concept Demonstration Plan}

What is the main risk, or risks, for the success of your project?  What will you
demonstrate during your proof of concept demonstration to convince yourself that
you will be able to overcome this risk?

\section{Technology}

\begin{itemize}
\item Specific programming language
\item Specific linter tool (if appropriate)
\item Specific unit testing framework
\item Investigation of code coverage measuring tools
\item Specific plans for Continuous Integration (CI), or an explanation that CI
  is not being done
\item Specific performance measuring tools (like Valgrind), if
  appropriate
\item Libraries you will likely be using?
\item Tools you will likely be using?
\end{itemize}

\section{Coding Standard}

Code written should conform to the official rust style guideline, found at: \url{https://doc.rust-lang.org/1.0.0/style/README.html} and to google's JavaScript style guide, found at: \url{https://google.github.io/styleguide/jsguide.html}.\\
Classes and methods are to be named using $PascalCase$, and variables and parameters are to be named using $camelCase$.
\newline
The goal is to maintain readability and ease of maintenance for any other developers.

\section{Project Scheduling}
\subsection{Scheduling software}
The project schedule is organized as epics and issues in the team's Zenhub board. The epics are based of major milestones such as POC, Revision 0, Revision 1, etc. The issues are various tasks associated to these epics.
\subsection{Task breakdown}
The big tasks are broken down to issues which can be completed by the individual developers in 1-5 days any bigger tasks will be needed to be broken down. There will be some tasks such as the documents which will be worked on by multiple developers at once.
\subsection{Task assignment}
The developer assigned to each of the task will be assigned based on their proficiency related to the task, interest and available capacity. For example, one of our developer is extremely proficient at working with front end, thus will work on tasks related to front end. Some developers that are really interested in gaining experience in certain subjects will be assigned tasks related to that subject. Lastly, leftovers tasks will be assigned to developers who have leftover capacity.
\end{document}